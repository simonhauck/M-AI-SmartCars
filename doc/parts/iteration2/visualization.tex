\documentclass[.../Dokumentation.tex]{subfiles}
\begin{document}
\subsection{Darstellung}\label{sec-ita2-visualization}
Die, während \ref{sec-ita1-visualization} erdachte, Umsetzung einer abstrakteren 
Darstellungsweise wurde in diesem Arbeitsschritt hinsichtlich ihrer Umsetzung 
vorangetrieben.\\
Anhand eines ersten Papierprototypen, zu sehen in Abbildung 
\ref{fig-tree-paper}, wurde begonnen, das bereits bei der Herstellung der 
Fahrzeugprototypen gewonnene Wissen auf die Schaffung eines druckfähigen 
Baummodels anzuwenden.
\begin{figure}[H]
\begin{center}
    \includegraphics[
        width=0.5\linewidth,
    ]{imgs/tree_paper.jpg}
    \caption{Papierprototyp des Baums}
    \label{fig-tree-paper}
\end{center}
\end{figure}
\noindent
Mit den Maßen der Servomotoren als Referenz wurde mit den Modellierungsarbeiten 
begonnen. Eben diese Maße führten jedoch dazu, dass die geplante Umsetzung 
des Baums zu verhältnismäßig dicken Ästen führt. 
Die daraus resultierenden Dimensionen, insbesonder in die Tiefe, zeigt 
Abbildung \ref{fig-tree-side}.
\begin{figure}[H]
\begin{center}
    \includegraphics[
        width=0.5\linewidth,
    ]{imgs/tree_side.jpg}
    \caption{Dimensionen des Baummodels}
    \label{fig-tree-side}
\end{center}
\end{figure}
\end{document}