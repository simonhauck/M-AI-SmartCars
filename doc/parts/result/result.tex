% !TeX spellcheck = de_DE
\documentclass[.../Dokumentation.tex]{subfile}
\begin{document}
    \section{Fazit}\label{sec-result}
   	%TODO Image%
   	Der im Zuge der vier genannten Iterationen entwickelte und geschaffene Prototyp wird nachfolgend mit Bezug auf die Projektidee bewertet. Mit den beiden Fahrzeugen ist eine anfassbare Schnittstelle zum System realisiert worden. Die Fahrzeuge sind mittels 3D-Druck hergestellt und mit je einem Arduino ausgestattet. Hierdurch ist die drahtlose Kommunikation mit dem dahinterstehenden System möglich. Die Kombination aus der verwendeten Hardware und dem Herstellungsprozess der Fahrzeuge ermöglicht es den gewünschten Formfaktor zu erreichen. Trotz der verbauten Hardware sind die Fahrzeuge leicht und können auch von Kindern in eine Hand genommen und bedient werden. Zur Inbetriebnahme muss lediglich die Stromversorgung der Fahrzeuge mittels eines Schalters hergestellt werden. Die Arduinos sind so programmiert und konfiguriert, dass sie automatisch die Kommunikation mit dem dahinterliegenden System aufnehmen. \\\\
   	Dieses basiert auf dem Raspberry Pi 4 und soll den Umwelteinfluss der Fahrzeuge dem Nutzer darstellen. Obwohl die ursprüngliche Idee der Verwendung eines Displays verworfen werden musste, wird eine gleichwertige Darstellung auf Basis eines beweglichen Baums erzielt. Auch wenn es so nicht mehr möglich ist genaue Messwerte anzuzeigen, wird der Umwelteinfluss verschiedener Fahrzeugtypen deutlich. Dieser Eindruck wurde durch den Kontakt mit projektfremden Personen, unter anderem im \emph{FabLab}, bestätigt. Wie zuvor bei den Fahrzeugen auch, muss für den Betrieb lediglich der Raspberry Pi mit Strom versorgt werden.\\\\
   	Abschließend lässt sich festhalten, dass die zu Beginn formulierte Idee trotz angefallener Änderungen erfolgreich umgesetzt werden konnte. Die Komponenten des Gesamtsystems interagieren reibungslos miteinander und können den gewünschten Lerneffekt beziehungsweise damit eine Verhaltensänderung bieten.
   	
\end{document}