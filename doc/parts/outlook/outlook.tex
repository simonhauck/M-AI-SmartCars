% !TeX spellcheck = de_DE
\documentclass[.../Dokumentation.tex]{subfile}
\begin{document}
\subsection{Ausblick}\label{sec-outlook}
Da im Rahmen der Veranstaltung auf Grund der begrenzten Zeit und Ressourcen nur die Entwicklung eines einfachen Prototypen möglich ist, werden im Folgenden etwaige Erweiterungen und Optimierungen vorgestellt.
\begin{itemize}
	\item \textbf{Anpassung des SUV-Modells}:\\
	Das erste SUV-Fahrzeugmodell diente nur zum Herstellen der Funktionalität. 
	Deshalb wurde auf eine detailreiche Modellierung verzichtet. 
	Vor einem weiteren Druck des Modells sollte es überarbeitet werden.
	\item \textbf{Weitere Fahrzeugtypen}:\\
	Da die für die Funktion erforderliche Elektronik einen fest definierten 
	Platzbedarf hat, können weitere Fahrzeuge einfach erstellt werden. Hierzu 
	muss lediglich die äußere Form angepasst und auf dem Arduino der 
	Fahrzeugtyp hinterlegt werden. 
	\item \textbf{Baum}:\\
	Für den Baum wurde eine Vorlage aus dem Internet verwendet. Für einen 
	langfristigen Einsatz sollte eine auf den Anwendungskontext zugeschnittene 
	Version erstellt werden.
	\item \textbf{Kiste für den Baum}:\\
	Für eine einfachere Montage und ein stimmigeres Gesamtbild kann die Kiste 
	mit Aussparungen und Einbauhilfen für die Komponenten ausgestattet werden. 
	Somit sind keine manuellen Anpassungen mehr nötig.
	\item \textbf{Display}:\\
	Mit weiteren finanziellen Mitteln kann die Darstellung, entsprechend dem 
	ursprünglichen Konzept, um ein Display erweitert werden.
	\item \textbf{Unterlage}:\\
	Mit dem Hall-Sensor kann die gefahrene Strecke sehr genau ermittelt werden. 
	Stünden eine maßstabsgetreue Karte als Unterlage und das oben genannte 
	Display zur Verfügung, könnten gefahrene Strecken und deren Emissionen 
	genau berechnet und abgebildet werden.
	\item \textbf{Platine}:\\
	Da die manuelle Fertigung der Platinen sehr aufwändig ist, sollten diese 
	bei größeren Produktionsmengen professionell hergestellt werden.
\end{itemize}
   
\end{document}