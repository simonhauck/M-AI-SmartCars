% !TeX spellcheck = de_DE
\documentclass[.../Dokumentation.tex]{subfiles}
\begin{document}
\section{Konzept}\label{sec-concept}
Im Folgenden soll nun umrissen werden, wie wir unsere in \ref{sec-intr-idea} 
formulierte Idee realisieren wollen.\\
Als Kontaktpunkt des Nutzers mit dem System sehen wir ein oder mehrere der 
erwähnten Fahrzeuge. Diese sollen händisch bewegt werden. Im Gegensatz hierzu 
stünde eine Umsetzung, in der die Fahrzeuge zum Beispiel mit Hilfe einer 
App fern gesteuert werden können.
Hiervon erhoffen wir uns zum Einen eine einfachere, vor allem aber auch 
kostengünstigere Umsetzung und zum Anderen ein intuitiveres Bedienerlebnis\\
Um hierauf aufzubauen und den thematischen Schwerpunkt des Klimapakt Flensburg 
aufzugreifen, soll die Möglichkeit in Betracht gezogen werden, anstelle 
eines generischen Teppichs eine maßstabsgetreue Darstellung der Flensburger 
Innenstadt sowie eventuell des Umlands zu verwenden.
Auf diesem Wege wäre es weiter möglich, die Konsequenzen auch kleinerer Fahrten 
zu verdeutlichen.\\
Die Darstellung mittels des Displays könnte neben rohen Zahlen weiter um eine 
Repräsentation der Emissionen in weniger abstrakter Form, wie eine Menge 
an bestimmten Lebensmitteln, deren Produktion und Transport gleiche Werte 
verursachen, ergänzt werden.
Weiter könnte das Display um eine zusätzliche Luftpumpe ergänzt werden, 
welche Ballons entsprechend der Emissionen aufpumpt.
\end{document}