\documentclass[.../Dokumentation.tex]{subfiles}
\begin{document}
\subsection{Umsetzung}\label{sec-concept-execution}
Um die Anforderungen an die Fahrzeuge erfüllen zu können, wurde 
einerseits evaluiert, fertige Spielzeugautos zu erwerben und um die 
elektronischen Komponenten zu erweitern, die zur Messung der zurück gelegten 
Distanz und der Kommunikation benötigt werden. Allerdings hätte sich vorab 
nicht ohne Zweifel sagen lassen können, ob der für diese Erweiterungen 
erforderliche Platz vorhanden ist oder sich schaffen lässt, ohne das Auto 
seiner Funktion zu berauben. 
Andererseits bestand die Option, die Fahrzeuge von 
Grund auf selbst herzustellen. 
Auf diese Weise kann sichergestellt werden, 
dass genug Platz für die erforderliche Technik vorhanden ist.
An dieser Stelle ensteht jedoch ein Konflikt zwischen dem Platzbedarf der zu 
ermittelnden Komponenten und 
einem der Projektidee entsprechenden Formfaktor der Fahrzeuge.
Weiter sollen die Fahrzeuge in einer Form geschaffen werden, die es erlaubt, 
ohne weitere Konfiguration "los zu fahren", sobald eine initiale, 
quasi werksseitige, Konfiguration durchgeführt wurde und die Stromversorgung 
hergestellt ist. 
Die Stromversorgung der Fahrzeuge soll mit Hilfe von Akkus umgesetzt werden, 
damit keine Kabel etwaiger Netzteile bei den Bewegungen der Fahrzeuge 
berücksichtigt werden müssen oder diese gar behindern.
Ein Verzicht auf Kabel soll auch in der Kommunikation der Fahrzeuge mit 
der Steuerung der Darstellung gewahrt bleiben.
Hierdurch entsteht die Anforderung, beide Seiten des Systems mit solchen 
Bauteilen auszustatten, die eine drahtlose Kommunikation erlauben.
Die erforderlichen Komponenten sollen so gewählt und installiert werden, 
dass eine Wartung ohne unverhältnismäßigen Aufwand möglich ist. Weiter soll 
durch die Durchführung der Montage gewährleistet werden, dass kostspieligere 
Bauteile entfernt und anderweitig erneut eingesetzt werden können.
\end{document}