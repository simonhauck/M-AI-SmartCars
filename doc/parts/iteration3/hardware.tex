% !TeX spellcheck = de_DE
\documentclass[.../Dokumentation.tex]{subfiles}
\begin{document}
    \subsection{Hard- und Software}
    \label{sec-ita3-hardware}
    \subsubsection*{Visualisierung}
    Die erste funktionsfähige Fahrzeugsoftware bildet die Grundlage für die Implementation der Visualisierung. Die Basis hierfür ist das Webframework \emph{Flask}. Mit diesem wird eine REST-Schnittstelle entworfen, welche die gesendeten Fahrzeugdaten empfängt. \\
    In einer Konfigurationsdatei werden die verschiedenen Fahrzeugtypen und die zugehörigen Fahrzeugeigenschaften gespeichert. Die Eigenschaften umfassen eine eindeutige ID, den Namen des Typs, den Radius der Räder, die Anzahl an Magneten in diesen und die verursachte Verschmutzung pro gefahrenem Zentimeter. Somit muss der Programmcode auch bei zukünftigen Änderungen nicht angepasst werden und verschiedene Fahrzeugtypen können einfach eingefügt und abgebildet werden.\\
    Wenn ein Fahrzeug sich bewegt, sendet es die Anzahl an erfassten Magnetkontakten und die ID des Fahrzeugs an den Raspberry Pi. Dieser berechnet im ersten Schritt mithilfe der spezifizierten Werten in der Konfigurationsdatei die verursachte Verschmutzung. Das Ergebnis wird im zweiten Schritt mit einem Zeitstempel versehen und gespeichert. Da die Messwerte nur wenige Sekunden aktuell sind und keine persistente Speicherung benötigen, wird keine Datenbank aufgesetzt. Die Werte werden lediglich im RAM gespeichert.\\
    Das Programm startet zu Beginn einen weiteren Thread. Dieser ist für die Steuerung der Hardwarekomponenten verantwortlich. Hierfür wird alle $x$ Sekunden eine Funktion aufgerufen. Diese iteriert zuerst über alle gespeicherten Verschmutzungseinträge und löscht die veralteten. Hierfür wird die aktuelle Systemzeit und der Zeitstempel des Eintrags verglichen. Die Gültigkeit von diesen kann in der Konfigurationsdatei angepasst werden. Danach werden die verbleibenden gültigen Werte addiert und ergeben die gesamte Verschmutzung. Diese wird an die einzelnen Hardware Komponenten weitergeleitet.\\
    Für die Steuerung der Servo Motoren wird ein PWM Signal verwendet. Mit der zuvor berechneten Verschmutzung wird die Rotation von diesen bestimmt. Dabei entspricht $0 \degree$ keiner und $180 \degree$ der maximalen Verschmutzung.\\
    Für die LEDs wird der WS2812B LED-Streifen von der Firma Adafruit verwendet. Dieser wird über die dazugehörige Python Bibliothek angesteuert. Je nach Verschmutzung setzt sich die Farbe aus Grün und Rot zusammen.\\
    Die hierfür verwendet Schaltung ist in Abbildung xy dargestellt. 
    %TODO Bild einfpgen%
   	Diese enthält zusätzlich einen Kondensator und eine Schnittstelle für eine externe Stromversorgung. Der Kondensator soll Schwankungen der Spannung ausgleichen. Das Netzteil wird benötigt um den LED-Streifen mit ausreichend Strom zu versorgen, da der Raspberry Pi nicht genug liefert würde. Mit dieser Schaltung kann das externen Netzteil zusätzlich den Raspberry Pi über den 5V mit Strom versorgt werden. Somit wird nur ein Netzeil für alle Komponenten  benötigt.
   	
   	\subsubsection*{Fahrzeug}
   	Beim Testen der Visualisierung mit der Fahrzeugschaltung ist ein Problem aufgetreten. Ist das Fahrzeug mit einem USB Kabel an den PC angeschlossen funktioniert die Sensorerkennung. Wird diese nur mit dem Akku betrieben, können keine Sensorwerte erkannt werden.\\
   	Beim Ausmessen der verfügbaren Spannung zeigt sich, dass im Betrieb mit dem Akku nur $3.3V$ an dem $5V$ Pin verfügbar sind. Wie in Kapitel \ref{sec-ita1-hardware} beschrieben, benötigt der Hall-Sensor mindestens $3.7V$.\\
   	Zum lösen dieses Problems muss die Spannung angehoben werden, was kein triviales Problem ist. Der erste Lösungsansatz ist ein \emph{Logic-Level-Converter}, welcher zwischen dem Sensor und dem Arduino eingebaut werden kann. Dieser kann Signale in bidirektionaler Richtung zwischen zwei Spannungen, wie zum Beispiel $5V \longleftrightarrow 3.3V$ wandeln. Das Problem hierbei ist, dass das Modul an einem Pin eine $5$ Spannung angelegt haben muss. Diese ist nicht verfügbar.\\
   	Der zweit Ansatz ist das verwenden eines $Step-Up Boost-Converter$, mit welchem das Problem erfolgreich gelöst werden kann. Das inAbbildung xy dargestellte Modul wird in diesem Fall verwendet.   	%TODO Abbildung%
   	Dieses akzeptiert eine Eingangsspannung zwischen $2-24V$ und hat eine Ausgangsspannung von $5-28V$. Die Ausgangsspannung kann mit einem Potentiometer eingestellt werden. Die Schaltung des Fahrzeugs wird mit diesem Modul abgeändert. Die neue Version ist in Abbildung xy dargestellt. %TODO Abbildung%
   	Der $+$ und $GND$ Pin des Sensors werden an den Output des Potentiometer angeschlossen, welches auf $5V$ eingestellt wird. Das Potentiometer erhält die benötigte Eingangsspannung vom $3.3V$ Pin des Arduinos. Somit funktioniert die Schaltung sowohl im Betrieb mit einem USB Kabel als auch mit einem Akku. 
    
    
\end{document}