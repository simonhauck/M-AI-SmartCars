\documentclass[.../Dokumentation.tex]{subfiles}
\begin{document}
\subsection{Fahrzeuge}\label{sec-ita3-cars}
Um alle Komponenten so einfach wie möglich montieren zu können, wurden 
weitere Anpassungen am Modell vorgenommen. 
Zunächst wurden im Frontbereich Aussparungen für die LED-\grqq 
Scheinwerfer\grqq{} geschaffen. Hierzu wurde der Durchmesser der LEDs 
aufgenommen und in Form von zwei \grqq Tunneln\grqq{} durch das Modell 
eingearbeitet. Mit dem Ziel einen sichereren Halt zu realisiern, wurde dieser 
Tunnel in etwa hinter der Länge der LEDs verengt. Auf diesem Wege soll eine 
LED an ihrem Platz bleiben können, während die erforderlichen Kabel trotzdem 
weiter in das Innere des Fahrzeugs geführt werden können.
Nach demselben Prinzip wurde darüber hinaus ein Platz zur Montage des 
Hall-Sensors im Radkasten geschaffen. Da keine genauen Maße zur Verfügung 
standen, wurde der Platzbedarf näherungsweise per Hand bestimmt und etwas 
erweitert, um etwaigen Messfehlern vorzubeugen.
Da sich das Problem um die übereifrige Platzierung von Stützmaterial durch 
\textit{Cura} in den vorherigen Iterationen nicht umgehen ließ, sollte 
dieser Stand des Modells nun zu Evaluationszwecken gänzlich ohne Stützmaterial 
gedruckt werden.
\end{document}