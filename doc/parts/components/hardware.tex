% !TeX spellcheck = de_DE
\documentclass[.../Dokumentation.tex]{subfiles}
\begin{document}
    \subsection{Hardware}\label{sec-components-hardware}
    Das Projekt besteht aus zwei Teilen, den Fahrzeugen und der Visualisierung, für welche die benötigte Hardware in diesem Kapitel vorgestellt wird. Die vorgestelltem Komponenten beziehen sich dabei auf das Konzept und ändern sich im Verlauf des Projektes.\\~\\
    \subsubsection{Fahrzeuge}
	Die Hardware für die Fahrzeuge wird mit folgenden Anforderungen ausgewählt: Das Fahrzeug soll klein und kompakt sein, sodass auch kleine Kinder mit diesem spielen können. Die Elektronik des Fahrzeugs soll über einen wiederaufladbaren Akku betrieben werden. Somit ist kein ständiger Batteriewechsel nötig und es behindern keine Kabel das Spielvergnügen. Das Fahrzeug soll die gefahrene Distanz selbstständig erfassen können und hierfür keine externe Hardware, wie zum Beispiel Kameras, benötigen. Der Aufbau soll unkompliziert und schnell sein. Insgesamt soll der Prototyp eine Plug\&Play Erfahrung bieten.\\
	Dies Basis für die Fahrzeuge sollen Microcontroller sein. Diese benötigen sehr wenig Strom und ermöglichen so ein längeres Spielvergnügen. Für die Fahrzeuge soll der Arduino MKR Wifi 1010 verwendet werden. Dieser ist ca. 62mm lang und 25mm breit und somit bedeutend kleiner als der Arduino Uno. Zusätzlich ist der Arduino besonders for IoT Projekte geeignet, da ein Low Power Prozessor (SAMD21) verwendet wird. Das Board besitzt zudem eine integrierte Ladeschaltung und einen Anschluss für einen Lithium polymer Akku. Hierfür wird ein Modell mit einer Kapazität von 1000mAh verwendet, welcher einen Kompromiss zwischen Größe des Akkus und dem Formfaktor bietet. Des weiteren besitzt der verwendete Arduino ein interagiertes Wifi Modul, welches zu späteren Kommunikation mit der Visualisierung verwendet wird. \\
	Neben dem Microcontroller wird jedes Fahrzeug mit ein paar Magneten und einem Hall Sensor ausgestattet. Der Hall Sensor gibt ein Signal an den Arduino, wenn ein Magnet sich in dessen Nähe befindet. Diese werden später zur Erkennung von Radumdrehungen verwendet. Zuletzt werden zwei LEDs mit den dazugehörigen Widerständen als Scheinwerfer verwendet.
	
	\subsubsection{Visualisierung}
	
\end{document}