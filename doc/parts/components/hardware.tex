% !TeX spellcheck = de_DE
\documentclass[.../Dokumentation.tex]{subfiles}
\begin{document}
\subsection{Hardware}\label{sec-components-hardware}
Das Projekt besteht aus zwei Teilen, den Fahrzeugen und der Visualisierung.
Die hierfür benötigte Hardware soll im Folgenden vorgestellt werden.
Die genannten Komponenten beziehen sich auf das Konzept, welches in Abschnitt 
\ref{sec-concept} vorgestellt wurde.
\subsubsection{Fahrzeuge}
Die Basis für die Fahrzeuge sollen Microcontroller sein. Diese benötigen sehr 
wenig Strom und ermöglichen so ein längeres Spielvergnügen. 
Für die Fahrzeuge soll der Arduino MKR Wifi 1010 verwendet werden. Dieser ist 
ca. 62mm lang und 25mm breit und somit bedeutend kleiner als der Arduino Uno. 
Zusätzlich ist der Arduino besonders for IoT Projekte geeignet, da ein Low 
Power Prozessor (SAMD21) verwendet wird. Das Board besitzt zudem eine 
integrierte Ladeschaltung und einen Anschluss für einen Lithium-Polymer-Akku. 
Hierfür wird ein Modell mit einer Kapazität von 1000mAh verwendet, welches 
einen Kompromiss zwischen Größe des Akkus und dem Formfaktor bietet. Des 
Weiteren besitzt der verwendete Arduino ein integriertes Wifi Modul, welches 
später zur Kommunikation mit der Visualisierung verwendet wird. \\
Neben dem Microcontroller wird jedes Fahrzeug mit einem oder mehreren Magneten sowie einem 
Hall-Sensor ausgestattet. Der Hall-Sensor gibt ein Signal an den Arduino, 
wenn ein Magnet sich in dessen Nähe befindet. Diese werden später zur Erkennung 
von Radumdrehungen verwendet. Zuletzt werden zwei LEDs mit den dazugehörigen 
Widerständen als Scheinwerfer verwendet.

\subsubsection{Visualisierung}
Die Basis für die Visualisierung soll ein Raspbbery Pi 4 sein. An diesen kann 
ein Display angeschlossen werden, welches später die erfassten Daten 
aufbereitet anzeigt. Der Raspberry Pi fungiert zusätzlich als WLAN 
\textit{Access Point}. Die in den Fahrzeugen verwendeten Arduinos können sich 
direkt mit diesem verbinden und somit ist die kabellose Übertragung der Daten 
gewährleistet. Dieser Aufbau hat den Vorteil, dass keine weitere 
Hardware, wie zum Beispiel ein Router zwischen den Komponenten, benötigt wird.

\end{document}