% !TeX spellcheck = de_DE
\documentclass[.../Dokumentation.tex]{subfiles}
\begin{document}
\subsection{Software}\label{sec-components-software}
Für das Projekt werden zwei verschiedene Programmiersprachen verwendet. 
Die Arduinos verwenden standardmäßig eine C- beziehungsweise C++-ähnliche 
Programmiersprache, für welche die Entwicklungsumgebung 
\textit{Arduino IDE} verwendet wird. 
Für die Visualisierung stehen verschiedene Programmiersprachen zur Auswahl. 
Für dieses Projekt soll Python 3.7 verwendet werden. Mit dieser Sprache wird das einfache Erstellen eines Webservers, welcher zum Empfangen 
der Daten genutzt wird und das Ansteuern der IO-Pins beziehungsweise des 
Displays vom Raspberry Pi ermöglicht.

\subsubsection{Modellierung und Slicing}\label{sec-components-software-model}
Um die Fahrzeuge mit Hilfe von 3D-Druck herstellen zu können, musste eine 
Software gefunden werden mit der die benötigten Modelle erschaffen und angepasst 
werden können. Zweierlei Lösungen wurden hierfür in Betracht gezogen.
Zum einen \textit{Blender}\footnote{https://www.blender.org/},
da das erfoderliche Dateiformat 
\textit{stl} nativ unterstützt wird und derartig weit verbreitete Open 
Source Lösungen oft eine entsprechend große Community vorweisen können.
Zum anderen \textit{SketchUp} \footnote{https://www.sketchup.com/de},
welches zwar nur mit verringertem 
Funktionsumfang frei verfügbar ist, aber besonders durch seine einfache 
Bedienung überzeugt. Die Wahl fiel darauf, zunächst erste Gehversuche mit \textit{Blender} 
zu unternehmen und im Fall größerer Schwierigkeiten in der Bedienung auf 
\textit{SketchUp} zurückzugreifen, da bei dieser Software auf bereits 
vorhandene Erfahrungen zurückgegriffen werden kann. 
Die Unterstützung für das \textit{stl}-Format 
muss in diesem Fall jedoch über Erweiterungen sichergestellt werden. Zur Vorbereitung etwaiger Modelle für den Druck war es weiter nötig, 
eine geeignete Software für den \textit{Slicing}-Vorgang zu finden.
Nach erfolgter Recherche und auf Anraten des \textit{FabLab} der Hochschule 
Flensburg fiel die Wahl auf 
\textit{Cura}\footnote{https://ultimaker.com/de/software/ultimaker-cura}.
Neben einfacher Handhabung überzeugt \textit{Cura} durch die Unterstützung 
der zur Durchführung des Projekts verfügbaren Drucker und dem 
\textit{stl}-Format.
\end{document}