% !TeX spellcheck = de_DE
\documentclass[.../Dokumentation.tex]{subfiles}
\begin{document}
    \subsection{Software}\label{sec-components-software}
    Für das Projekt werden zwei verschiedene Programmiersprachen verwendet. 
    Die Arduinos verwenden standardmäßig eine C- beziehungsweise C++-ähnliche 
    Programmiersprache, für welche die Entwicklungsumgebung 
    \textit{Arduino IDE} verwendet wird. 
    Für die Visualisierung stehen verschiedene Programmiersprachen zur Auswahl. 
    Für dieses Projekt soll Python 3.7 verwendet werden. Diese Sprache 
    ermöglicht das einfache Erstellen eines Webservers, welcher zum Empfangen 
    der Daten genutzt wird und das Ansteuern der IO-Pins beziehungsweise des 
    Displays von dem Raspberry Pis.
    
    \subsubsection{Modellierung und Slicing}\label{sec-components-software-model}
    Um die für die Herstellung der Fahrzeuge mit Hilfe von 3D-Druck benötigten 
    Modelle anfertigen zu können, war es erforderlich, eine geeignete Software 
    zur Modellierung auszuwählen.\\
    Zweierlei Lösungen wurden hierfür in Betracht gezogen.
    Zum Einen \textit{Blender}, %TODO SOURCE
    da das nach eingangs erfolgter Recherche erfoderliche Dateiformat 
    \textit{stl} nativ unterstützt wird und derartig weit verbreitete Open 
    Source Lösungen oft eine entsprechend große Community vorweisen können.
    Zum Anderen \textit{SketchUp}, %TODO SOURCE
    welches zwar nur mit verringertem 
    Funktionsumfang frei verfügbar ist, aber besonders durch seine einfache 
    Bedienung überzeugen kann.\\
    Die Wahl fiel darauf, zunächst erste Gehversuche mit \textit{Blender} 
    zu unternehmen und im Fall größerer Schwierigkeiten in der Bedienung auf 
    \textit{SketchUp} zurück zu greifen, da mit dieser Software bereits 
    Erfahrungen gesammelt wurden. Die Unterstützung für das \textit{stl} Format 
    muss in diesem Fall jedoch über Erweiterungen sichergestellt werden.\\
    Zur Vorbereitung etwaiger Modelle für den Druck war es weiter nötig, 
    eine geeignete Software für den \textit{Slicing}-Vorgang zu finden.
    Nach erfolgter Recherche und auf Anraten des \textit{FabLab} der Hochschule 
    Flensburg fiel die Wahl auf \textit{Cura}. %TODO SOURCE
    Neben einfacher Handhabung überzeugt \textit{Cura} durch die Unterstützung 
    der zur Durchführung des Projekts verfügbaren Drucker und dem \textit{stl}-
    Format.
\end{document}