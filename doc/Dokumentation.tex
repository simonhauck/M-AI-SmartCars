\documentclass[a4paper, 11pt]{article}
\usepackage{float}
\usepackage{amsmath,amssymb,amsfonts}
\usepackage{algorithmic}
\usepackage{graphicx}
\usepackage{textcomp}
\usepackage{xcolor}
\usepackage{float}
\usepackage{amsthm}
\usepackage{geometry} 
\usepackage{pgfplots}
\usepackage{tikz}
\usepackage{listings}
\usepackage{authblk}
\usepackage{import}
\usepackage{subfiles}
\usepackage{hyperref}
\usepackage[ngerman]{babel}
\usepackage{gensymb}

\lstloadlanguages{Python}
\definecolor{bluekeywords}{rgb}{0,0,1}
\definecolor{greencomments}{rgb}{0,0.5,0}
\definecolor{redstrings}{rgb}{0.64,0.08,0.08}
\definecolor{xmlcomments}{rgb}{0.5,0.5,0.5}
\definecolor{types}{rgb}{0.17,0.57,0.68}

\lstset{language=Python,
    captionpos=b,
    frame=lines,
    showspaces=false,
    showtabs=false,
    breaklines=true,
    showstringspaces=false,
    breakatwhitespace=true,
    escapeinside={(*@}{@*)},
    commentstyle=\color{greencomments},
    morekeywords={partial, var, value, get, set},
    keywordstyle=\color{bluekeywords},
    stringstyle=\color{redstrings},
    basicstyle=\ttfamily\small,
}
\begin{document}
\pagenumbering{arabic}
\begin{titlepage}
    \begin{center}
        \vspace{3cm}
        {\LARGE\bfseries Hochschule Flensburg \par}
        \vspace{1.5cm}
        {\huge\bfseries SmartCars\par}
        \vspace{0.5cm}
        {\LARGE\bfseries Maker's Lab - Things that Think \par}
        \vspace{0.25cm}
        {\LARGE Abschlussbericht\par}
        \vspace{2cm}
        {\Large vorgelegt von: \par}
        \vspace{1cm}
        {\bfseries Simon Hauck \hfill \bfseries Nils Jensen\par}
        { 660158 \hfill  670758\par}
        { Hochschule Flensburg \hfill Hochschule Flensburg \par}
        { simon.hauck@stud.hs-flensburg.de \hfill  nils.jensen2@stud.hs-flensburg.de\par}
        \vfill
        \vfill
    % Bottom of the page
        {\large \today\par}
    \end{center}
\end{titlepage}
    \clearpage
    \thispagestyle{empty}
    \tableofcontents
    \pagebreak
    % Einleitung
    \subfile{parts/introduction/introduction.tex}    
        % Motivation
        \subfile{parts/introduction/motivation.tex}
        % Projektidee (Schwammprototyp)
        \subfile{parts/introduction/idea.tex}
    \pagebreak
    % Konzept
    \subfile{parts/concept/concept.tex}
        % Umsetzung
        \subfile{parts/concept/execution.tex}
        % etc
        % TODO maybe
    \pagebreak
    % Komponenten
    \subfile{parts/components/components.tex}
        % Hardware
        \subfile{parts/components/hardware.tex}
        % Software
        \subfile{parts/components/software.tex}
        % Werkzeuge
        \subfile{parts/components/tools.tex}
    % 1. Iteration                          flieder
    \subfile{parts/iteration1/iteration.tex}
        % Auto
        \subfile{parts/iteration1/cars.tex}
        % Elektronik
        \subfile{parts/iteration1/hardware.tex}
        % Software
        %\subfile{parts/iteration1/software.tex}
        % Darstellung 
        \subfile{parts/iteration1/visualization.tex}
        % Lessons Learned / Ergebnis
        \subfile{parts/iteration1/iteration_result.tex}
    % 2. Iteration                          transparent
        \subfile{parts/iteration2/iteration.tex}
        % Auto
        \subfile{parts/iteration2/cars.tex}
        % Elektronik
        \subfile{parts/iteration2/hardware.tex}
        % Software
        %\subfile{parts/iteration2/software.tex}
        % Darstellung 
        \subfile{parts/iteration2/visualization.tex}
        % Lessons Learned / Ergebnis
        \subfile{parts/iteration2/iteration_result.tex}
    % 3. Iteration                          gelb
    \subfile{parts/iteration3/iteration.tex}
        % Auto
        \subfile{parts/iteration3/cars.tex}
        % Elektronik
        \subfile{parts/iteration3/hardware.tex}
        % Software
        %\subfile{parts/iteration3/software.tex}
        % Darstellung 
        \subfile{parts/iteration3/visualization.tex}
        % Lessons Learned / Ergebnis
        \subfile{parts/iteration3/iteration_result.tex}
    % 4. Iteration                          gemischt
    \subfile{parts/iteration4/iteration.tex}
        % Auto
        \subfile{parts/iteration4/cars.tex}
        % Elektronik
        \subfile{parts/iteration4/hardware.tex}
        % Software
        %\subfile{parts/iteration4/software.tex}
        % Darstellung 
        \subfile{parts/iteration4/visualization.tex}
        % Lessons Learned / Ergebnis
        \subfile{parts/iteration4/iteration_result.tex}
    
    % Ergebnis
    \newpage
    \subfile{parts/result/result.tex}
    % Ausblick 
    \subfile{parts/outlook/outlook.tex}
\end{document}